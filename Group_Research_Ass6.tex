\documentclass[]{report}


% Title Page
\title{}
\author{KAHIGIRIZA PETER WARREN		216002577		16/U/5173/PS\\
	AINEBYONA DONALD			216013093		16/U/2970/PS\\
	KYOBWEINE PRISCA			216002579		16/U/6489/PS\\
	KIZITO EZEKIEL				213012200		13/U/6970/EVE}


\begin{document}
\maketitle 

\begin{flushleft}
	\underline{\textbf{Problem Statement}}\linebreak
	
	The current nature of lecturer and student attendance at the college of computing is not effectively monitored. However, it’s just a few lecturers that seemingly consider this issue a much important factor of student performance. As a result, they carry attendance sheets on which the available students in lecture rooms can record proof of their presence in the lecture.\linebreak
	
	 These sheets include their names, student numbers, registration numbers and a section for signatures. Unfortunately this is inefficient since it’s the very “attendees” that include details of their course mates that are not actually present at the ongoing lecture. This kind of attendance tracking is totally incapable of being legitimate.\linebreak
	 
	  This is simply because these very sheets can also be lost. Additionally, there’s also cheating attendance for those that are deliberately absent which makes it unethical. The desired solution at hand is specifically meant to record time, attendance and punctuality of the students per given course unit. Conclusively, this will enable lecturers to monitor student performance and improve authenticity of data.\linebreak
	  
	  
	\underline{\textbf{Objectives}}\linebreak
	
	\underline{Main}
	\begin{enumerate}
		\item To overcome the present manual attendance norm of using sheets to record “attendees”.
		\item To enable use of an enhanced platform for tracking of attendance.
	\item	To get rid of shortcomings of keeping attendance data on sheets.
	\item	To improve authenticity of actual students present in the lecture rooms.
		\item To improve attendance and punctuality of students in lecture rooms.
		
		
	\end{enumerate}
	\underline{Specific}
	\begin{itemize}
		\item To design a platform that eases attendance tracking of students.
	\item 	To enable lectures monitor and track student attendance and performance.
		
		
	\end{itemize}
\underline{\textbf{Research Scope}} \linebreak

The scope of the study is set to introduce a possibility of automating a rather manual system of making regular attendances for all the lectures at the college of computing and  information sciences specifically the program of computer science. The main focus is to include a study of how the intended system is to benefit students of computer science program.\linebreak

The research has a couple of limitations which are to be acknowledged. It also seeks to provide minimal information as a way to introduce only what’s required given the time frame. This means information regarding biometric hardware  as well as presenting an actual biometric machine with its added costs, capacity and other unidentified expenses is to be undefined in this case.\linebreak

\textbf{\underline{Literature Review}
}\linebreak

Over the years, there has been many developments and theories coming up with regards to attendance which could have been at places of work such as financial institutions for example micro-finance agencies, banks, etc as well as school. [1] A great scholar once stated that the use of a biometric-enabled security layer in accounting systems is aimed at enhancing user authentication and reducing control risks. He also added, that originating in criminology, biometric technology has matured over the years with the application in diverse disciplines. However, its use in business  and accounting is still in its infancy, and many issues about its role in information systems security is unresolved.[2] It also should be known that the purpose of the study is to assess the feasibility  of fingerprint, iris and face recognition technologies, to identify unknowns(ghost students) and the risks associated with the use of biometrics in such a national identity scheme, and to make  recommendations for how come of these risks might be addressed should such a scheme proceed. There has been several and attempted use of this format which works in the sense that [3] the system is remotely operated and is powered by a local power source such as an outlet in an office building. An example is the Globe Telecom company which provides a wireless service and a kiosk equipped with an RFID recognition. It’s also characterized with a fingerprint scanner to verifies identity of its members, which works in tandem with an electronic card reader.\linebreak

Additionally, it should also be noticed that Radio Frequency Identification(RFID) based interaction is quite of great importance in everyday attendance transactions. This is looked at the point of view of enabling biometric authentication. This means that each and every kiosk (lecture room) must support fingerprint-based biometric authentication.\linebreak

\underline{\textbf{Justification Of The Research}}\linebreak

From the extensive research carried out, the main possible benefits can be described as below;\begin{itemize}
	\item Improving authenticity of attendance and availability of students in lecture rooms.
	\item	Minimizing or if not baring any possibilities of students faking attendance.
	\item	Getting rid of human errors such as loss of lecturer’s attendance lists as well as students wrongly spelling the names and also registration details.
	\item	Tracking a legitimate attendance and punctual history of all students’ availability during lectures.
	
	
\end{itemize}

\underline{\textbf{Research Methodology}}\linebreak

In this section, we seek to exhaust description of the tools, instruments, approaches, processes, techniques, algorithms as well as the data structures that were employed to carry out the research study., the nature of data collection used, analysis and synthesis implored, the design and logical work flow of what we think the study yielded and also the necessary possibilities of the implementation, testing and validation.\linebreak

In a way to implement the desired main objectives of this study, we are to use both quantitative and qualitative methodologies since this involves both obtaining numbers of students as well as lecturer’s, time frame on which students attend.\linebreak

In a way to implement the desired main objectives of this study, we are to use both quantitative and qualitative methodologies since this involves both obtaining numbers of students as well as lecturer’s, time frame on which students attend.\linebreak

 It is at this point that we shall assess the opinions and trends about the students’ attendance and punctuality as a whole. It’s from this conclusion about the sex of the Computer Science class that we shall be able to determine which days of the week are mostly attended based on the class sheets. This will be harmonized with the college database concerning lecture attendance.\linebreak
 
  Later in our research study, we will also carry out a survey by handing out questionnaires to a sample set of 20 students. This action is intended at gathering information on the students’ reasons for their various attendance patterns.
 From this, we expect information about the following:
 \begin{itemize}
 	\item 	The attendance statistics about day- mainly morning and afternoon (  i.e. during 10am – 1pm) lectures and evening lectures.
 	\item Turn up of students with respect to the double-session(2hours) and single-session(1hour) lectures.
 	\item 	The attendance ratio of female to male and vice versa.
 	
 \end{itemize}
We are also going to carry out one-on-one interviews with a number of students and try to find out what is most preferred with regards to studying during morning and midmorning hours as well as late evening lectures.\linebreak

The interview will also include the lecturers’ input, this is to identify and establish a conclusion in their point of view with regards to the above survey guidelines.\linebreak

The study will require the lectures to outline attendance statistics of the day class and evening class in terms of quantity and the factors that they think could be causing the differences in attendance.\linebreak

It is known that currently there is an issue of fake attendance whereby the students that attend a particular lecture sign in for the absentees. We strongly believe that this is unethical. With the said methodologies in that manner, we hope that we will have exhausted the required information and raw data that we need to help us understand and therefore in a better position to design and develop and web-based application that can correlate attendance of both students and the lecturers. This hence monitors consistencies of attendance in a much integral and secure way.\linebreak

\textbf{\underline{Expected Functional Requirements Of the Web-based Application}}\linebreak

Below are the requirements that we expect to define what the application system is set out to look like;
\begin{itemize}
	\item 	It shall comprise of an authentication service system most likely the RFID system. This is to lock out any unidentified access by outsiders. The system will be designed with the basic entry requirements such as the user-name and password thus enabling an easy coordination. For the first-time users, the system will automatically prompt registration for an account.
	\item	On the lecturer’s end, the system shall automatically enlist all the legitimate and available attendees of that lecture. It shall list each student with their details against their timestamps of when they entered the lecture room.
	\item	The system shall also comprise of the administrator’s platform most likely the head of department who can then monitor both the students and lecturer’s attendance. This is with regards to the weekly and personal layout.
	\item 	The application is also expected to have a search utility within the database that can easily outline all the days attended and missed by a particular student or lecture; should there be need to query them.
	\item 	Also, in the expound stages; we anticipate that the application shall be in position to send weekly or daily reports to both the lecturer (concerning all student’s attendance) and students (concerning their individual attendance per course unit) during on-going semester period.
	\item 	Finally, it will also offer summaries on to the administrator’s end.
	
\end{itemize}\linebreak




\textbf{\underline{Appendices}}\linebreak

DFD- Data Flow Diagrams\\
RFID- Radio Frequency Identification























\end{flushleft}


\end{document}          
